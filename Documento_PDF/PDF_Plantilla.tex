% Options for packages loaded elsewhere
\PassOptionsToPackage{unicode}{hyperref}
\PassOptionsToPackage{hyphens}{url}
\PassOptionsToPackage{dvipsnames,svgnames,x11names}{xcolor}
%
\documentclass[
  letterpaper,
  DIV=11,
  numbers=noendperiod]{scrartcl}

\usepackage{amsmath,amssymb}
\usepackage{iftex}
\ifPDFTeX
  \usepackage[T1]{fontenc}
  \usepackage[utf8]{inputenc}
  \usepackage{textcomp} % provide euro and other symbols
\else % if luatex or xetex
  \usepackage{unicode-math}
  \defaultfontfeatures{Scale=MatchLowercase}
  \defaultfontfeatures[\rmfamily]{Ligatures=TeX,Scale=1}
\fi
\usepackage{lmodern}
\ifPDFTeX\else  
    % xetex/luatex font selection
\fi
% Use upquote if available, for straight quotes in verbatim environments
\IfFileExists{upquote.sty}{\usepackage{upquote}}{}
\IfFileExists{microtype.sty}{% use microtype if available
  \usepackage[]{microtype}
  \UseMicrotypeSet[protrusion]{basicmath} % disable protrusion for tt fonts
}{}
\makeatletter
\@ifundefined{KOMAClassName}{% if non-KOMA class
  \IfFileExists{parskip.sty}{%
    \usepackage{parskip}
  }{% else
    \setlength{\parindent}{0pt}
    \setlength{\parskip}{6pt plus 2pt minus 1pt}}
}{% if KOMA class
  \KOMAoptions{parskip=half}}
\makeatother
\usepackage{xcolor}
\usepackage[left = 2.5cm,top = 2.0cm,right = 2.5cm]{geometry}
\setlength{\emergencystretch}{3em} % prevent overfull lines
\setcounter{secnumdepth}{-\maxdimen} % remove section numbering
% Make \paragraph and \subparagraph free-standing
\makeatletter
\ifx\paragraph\undefined\else
  \let\oldparagraph\paragraph
  \renewcommand{\paragraph}{
    \@ifstar
      \xxxParagraphStar
      \xxxParagraphNoStar
  }
  \newcommand{\xxxParagraphStar}[1]{\oldparagraph*{#1}\mbox{}}
  \newcommand{\xxxParagraphNoStar}[1]{\oldparagraph{#1}\mbox{}}
\fi
\ifx\subparagraph\undefined\else
  \let\oldsubparagraph\subparagraph
  \renewcommand{\subparagraph}{
    \@ifstar
      \xxxSubParagraphStar
      \xxxSubParagraphNoStar
  }
  \newcommand{\xxxSubParagraphStar}[1]{\oldsubparagraph*{#1}\mbox{}}
  \newcommand{\xxxSubParagraphNoStar}[1]{\oldsubparagraph{#1}\mbox{}}
\fi
\makeatother


\providecommand{\tightlist}{%
  \setlength{\itemsep}{0pt}\setlength{\parskip}{0pt}}\usepackage{longtable,booktabs,array}
\usepackage{calc} % for calculating minipage widths
% Correct order of tables after \paragraph or \subparagraph
\usepackage{etoolbox}
\makeatletter
\patchcmd\longtable{\par}{\if@noskipsec\mbox{}\fi\par}{}{}
\makeatother
% Allow footnotes in longtable head/foot
\IfFileExists{footnotehyper.sty}{\usepackage{footnotehyper}}{\usepackage{footnote}}
\makesavenoteenv{longtable}
\usepackage{graphicx}
\makeatletter
\def\maxwidth{\ifdim\Gin@nat@width>\linewidth\linewidth\else\Gin@nat@width\fi}
\def\maxheight{\ifdim\Gin@nat@height>\textheight\textheight\else\Gin@nat@height\fi}
\makeatother
% Scale images if necessary, so that they will not overflow the page
% margins by default, and it is still possible to overwrite the defaults
% using explicit options in \includegraphics[width, height, ...]{}
\setkeys{Gin}{width=\maxwidth,height=\maxheight,keepaspectratio}
% Set default figure placement to htbp
\makeatletter
\def\fps@figure{htbp}
\makeatother

\usepackage{booktabs}
\usepackage{longtable}
\usepackage{array}
\usepackage{multirow}
\usepackage{wrapfig}
\usepackage{float}
\usepackage{colortbl}
\usepackage{pdflscape}
\usepackage{tabu}
\usepackage{threeparttable}
\usepackage{threeparttablex}
\usepackage[normalem]{ulem}
\usepackage{makecell}
\usepackage{xcolor}
\KOMAoption{captions}{tableheading}
\usepackage{fancyhdr}
\usepackage{ragged2e}
\usepackage{geometry}
\usepackage{amsmath, amsfonts, amssymb, latexsym, amsthm, amsfonts}
\usepackage{float}
\usepackage{subcaption}
\usepackage[dvips]{psfrag}
\usepackage{enumerate}
\usepackage{graphicx}
\usepackage{caption}
\usepackage{multirow}  % Para tablas
\usepackage{flushend}
\usepackage{hyperref}
\makeatletter
\@ifpackageloaded{caption}{}{\usepackage{caption}}
\AtBeginDocument{%
\ifdefined\contentsname
  \renewcommand*\contentsname{Table of contents}
\else
  \newcommand\contentsname{Table of contents}
\fi
\ifdefined\listfigurename
  \renewcommand*\listfigurename{List of Figures}
\else
  \newcommand\listfigurename{List of Figures}
\fi
\ifdefined\listtablename
  \renewcommand*\listtablename{List of Tables}
\else
  \newcommand\listtablename{List of Tables}
\fi
\ifdefined\figurename
  \renewcommand*\figurename{Figure}
\else
  \newcommand\figurename{Figure}
\fi
\ifdefined\tablename
  \renewcommand*\tablename{Table}
\else
  \newcommand\tablename{Table}
\fi
}
\@ifpackageloaded{float}{}{\usepackage{float}}
\floatstyle{ruled}
\@ifundefined{c@chapter}{\newfloat{codelisting}{h}{lop}}{\newfloat{codelisting}{h}{lop}[chapter]}
\floatname{codelisting}{Listing}
\newcommand*\listoflistings{\listof{codelisting}{List of Listings}}
\makeatother
\makeatletter
\makeatother
\makeatletter
\@ifpackageloaded{caption}{}{\usepackage{caption}}
\@ifpackageloaded{subcaption}{}{\usepackage{subcaption}}
\makeatother

\ifLuaTeX
  \usepackage{selnolig}  % disable illegal ligatures
\fi
\usepackage{bookmark}

\IfFileExists{xurl.sty}{\usepackage{xurl}}{} % add URL line breaks if available
\urlstyle{same} % disable monospaced font for URLs
\hypersetup{
  colorlinks=true,
  linkcolor={blue},
  filecolor={Maroon},
  citecolor={Blue},
  urlcolor={Blue},
  pdfcreator={LaTeX via pandoc}}


\author{}
\date{2025-04-25}

\begin{document}


\renewcommand{\tablename}{Tabla}
\renewcommand{\figurename}{Figura}
\renewcommand{\refname}{References}

\renewcommand{\footrulewidth}{0.4pt}
\renewcommand{\headrulewidth}{0.4pt}
\hypersetup{
colorlinks=true,
linkcolor=blue,
filecolor=magenta,
urlcolor=cyan,
pdftitle={Proyecto de Módulo},
bookmarks=true,
pdfpagemode=FullScreen,
}

\pagestyle{fancy}
\fancyhead[R]{}
\fancyhead[L]{\footnotesize{Caso de estudios }}
\fancyfoot[L]{\includegraphics[height=0.5cm]{imagenes/logoc.png}}
\fancyfoot[R]{}

\title{
\LARGE
The Impact of SITM Vehicle Breakdowns on Economic Losses and Perceived Service Quality in Cartagena\\[0.5em]
\large % Cambia el tamaño aquí según lo que prefieras
Impacto de las varadas de vehículos del SITM en las pérdidas económicas y la percepción del servicio en Cartagena
}

\author{Alexander Sánchez González \thanks{ Universidad Tecnológica de
Bolívar, Colombia, e-mail: \url{sancheza@utb.edu.co}, ID: T00054514}
\and Amparo Hazbun Martinez \thanks{ Universidad Tecnológica de
Bolívar, Colombia, e-mail: \url{ahazbun@utb.edu.co}, ID: T000XXXXX}
\and Maria Mercedes Romero Racine \thanks{ Universidad Tecnológica de
Bolívar, Colombia, e-mail: \url{mracine@utb.edu.co}, ID: T000XXXXX}
\and Misael Jose Pastrana Fuentes \thanks{ Universidad Tecnológica de
Bolívar, Colombia, e-mail: \url{pastranam@utb.edu.co}, ID: T000XXXXX}
}

\begin{@twocolumnfalse}
\maketitle

\begin{center}\rule{12cm}{0.3mm} \end{center}

\begin{center}
\textbf{Abstract} 
\end{center}
 
\justify

\textbf{Background:}

\textbf{Aims:} 

General Objective:

To analyze the impact of vehicle breakdowns within the Integrated System of Mass Transit (SITM) of the city of Cartagena on the economic losses of the operating company and on the perceived service quality by users.

Specific Objectives:

\begin{enumerate} 
  \item Identify the routes and locations with the highest frequency and costs of breakdowns. 
  \item Determine the main operational causes associated with breakdowns and their relationship with user satisfaction.    \item Quantify the economic losses caused by vehicle breakdowns in the SITM, taking into account operational and maintenance costs, as well as unproductive time. 
\end{enumerate}

\textbf{Methods:}

Analysis of breakdown records, analysis of economic losses, assessment of impact by route and location, statistical analysis.

\textbf{Results:}

\textbf{Conclusions:}



\textbf{\textit{Keywords:}}
Stranded, Economic losses, Service quality, Users, SITM (Integrated System of Mass Transit).


\begin{center}
\textbf{Resumen} 
\end{center}

\justify


\textbf{Antecedentes:}

\textbf{Objetivos:}

Objetivo general:

Analizar el impacto de las varadas de los vehículos del Sistema Integrado de Transporte Masivo (SITM) de la ciudad de Cartagena en las pérdidas económicas de la empresa operadora y en la calidad del servicio percibida por los usuarios.

Objetivos específicos:

\begin{enumerate}
    \item Identificar las rutas y ubicaciones con mayor frecuencia y costos de varadas.
    \item Determinar las principales causas operativas asociadas a las varadas y su relación con la satisfacción del usuario.
    \item Cuantificar las pérdidas económicas generadas por las varadas de los vehículos del SITM, considerando costos operativos, de mantenimiento y los tiempos improductivos.
\end{enumerate}

\textbf{Método:}
Se realizó un estudio transversal descriptivo sobre una base  de fallas en vehículos del Sistema Integrado de Transporte Masivo (SITM) de la ciudad de Cartagena. El análisis abarcó desde el 1 de septiembre de 2024 hasta el 1 de agosto de 2025.  por su parte, para las variables cualitativas se estimaron frecuencias absolutas y relativas y en el caso de las variables cuantitativas estas fueron resumidas a partir del uso de medias de tendencia central y dispersión particular, se usaron medias,frecuencias relativas y absolutas. La estimación del costo de oportunidad fue calculada teniendo en cuenta el numero de personas maxima que los vehiculos pueden trasportar y multiplicado por el costo total de la tarija a 2024 que era de $3000 pesos.

\textbf{Resultados:}
\begin{enumerate}
    \item Durante los siete meses del estudio, se registraron un total de 3,635 fallas. Las rutas X106 (546 fallas) y X104 (524 fallas) presentaron el mayor número de incidencias. Asimismo, se observa que el tipo de vehículo con mayor frecuencia de averías es el Padron, con un total de 2,484 fallas, lo que representa el 68% del total.
    \item Se observa que las fallas ocurren principalmente en el Patio Portal, con un total de 1,176 incidentes (37%), seguido por la Estación Bodeguita con 655 fallas (20%) y la Terminal con 191 fallas (5.9%). Estas averías suelen presentarse entre las 5:30 p. m. y las 7:00 p. m. En cuanto a las causas, el problema más frecuente es la falla en el sistema de aire comprimido, con 292 casos (18%), seguido por altas temperaturas y riesgos de daños en componentes y motores, con 230 reportes (14%). Otra avería común es el mal funcionamiento de las puertas, que representa 158 casos (9.8%).
    \item En términos de costos, se identificó que la mediana de los costos por operación sugiere que el 50% de los costos de oportunidad mínimos son menores a 900,000, mientras que la otra mitad supera este valor. Por otro lado, el costo mínimo promedio de oportunidad calculado fue de 20,519,571 pesos colombianos por día varado, mientras que el máximo alcanzó los 43,889,084 pesos colombianos, lo que representaría una pérdida promedio de 1,545 usuarios.
\end{enumerate}

  
\textbf{Conclusión:}

En conclusión, la recurrencia de varadas y sus consecuencias evidencian la necesidad de **optimizar los procesos de mantenimiento**, implementar **estrategias de prevención de fallas** y considerar **posibles mejoras en la infraestructura vehicular**. Abordar estos aspectos no solo contribuiría a reducir los costos operativos, sino que también mejoraría la experiencia del usuario y la continuidad del servicio, fortaleciendo la eficiencia y sostenibilidad del SITM. esto dado que se identifico por medio del análisis de las varadas de los vehículos del **Sistema Integrado de Transporte Masivo (SITM)**, que estos generan un impacto significativo tanto en las **pérdidas económicas de la empresa operadora** como en la **calidad del servicio percibida por los usuarios**.  

dado que si lo miramos desde la perspectiva financiera, las interrupciones en el servicio generan costos elevados, con un costo mínimo promedio de **20,519,571 pesos colombianos por día varado**, alcanzando hasta **43,889,084 pesos** en los peores escenarios. Esto no solo representa un riesgo financiero considerable para la empresa, sino que también implica una pérdida diaria estimada de **1,545 usuarios**, afectando la sostenibilidad del sistema y la confianza en el usuario.Por su parte, en términos operativos, las fallas ocurren con mayor frecuencia en puntos clave como **el Patio Portal (37%)**, **la Estación Bodeguita (20%)** y **la Terminal (5.9%)**, y  estas suelen presentarse en horarios críticos entre **las 5:30 p. m** y **las 7:00 p. m**. Los problemas más recurrentes afectan a los buses son  **aire comprimido (32%)** y **el motor (23%)**, así como subsistemas esenciales como **los frenos (16%)** y **el sistema de refrigeración (16%)**.  

pero esta  alta incidencia  de fallas se presenta en los vehículos de tipologia **Padron (68%)** lo que sugiere que ciertos modelos requieren **mayor mantenimiento o actualización** para reducir la frecuencia de varadas y mejorar la confiabilidad del sistema. Además, las fallas en el **sistema de aire comprimido (18%)** y los riesgos derivados de **altas temperaturas en los componentes** y **el motor (14%)** destacan como factores clave que deben ser abordados para mejorar la eficiencia del servicio, reducir la afectación  al servicio y mejor aumentar las ganancias producto del servicio.  


\textbf{\textit{Palabras clave:}}
Varadas, Pérdidas económicas, Calidad del servicio, Usuarios, SITM (Sistema Integrado de Transporte Masivo).


\begin{center}\rule{12cm}{0.3mm} \end{center}
\end{@twocolumnfalse}

\subsection{1. Introduction}\label{introduction}

The Integrated System of Mass Transit (SITM) in the city of Cartagena
plays a fundamental role in urban mobility. Its proper functioning
largely depends on the condition of the fleet and the operational
efficiency of the vehicles. Consequently, one of the most recurring
challenges faced by these systems is vehicle breakdowns or operational
disruptions, which generate significant impacts for both the operating
company and the users.

These breakdowns entail operational and maintenance costs associated
with unproductive time. For users, such failures translate into delays,
discomfort, and a loss of trust in the system.

This study aims to comprehensively analyze these failures within the
SITM, starting with the identification of associated costs, as well as
the routes and locations where they occur most frequently, in order to
identify critical operational patterns.

Finally, the study seeks to quantify the economic losses resulting from
breakdowns, highlight the main areas of impact, and assess the negative
consequences for users, with the ultimate goal of enhancing transparency
and rebuilding trust in the system.

Based on this context, the following research question arises: What is
the impact of vehicle breakdowns in Cartagena's Integrated System of
Mass Transit (SITM) on the operating company's economic losses and on
the perceived quality of service by users?

\subsection{2. Data}\label{data}

The collected data corresponds to a representative sample of buses and
routes within the city's Integrated Mass Transit System (SITM);
therefore, it does not cover the system in its entirety. Additionally,
the analysis is limited to a seven-month period. The information was
provided by one of the system's operating companies. Both qualitative
and quantitative data were gathered, although the latter predominates.
Below is a detailed summary of the types and amounts of the final data
to be used in the analysis.

\begin{table}[H]
\centering
\caption{Number of Variables by Type}
\centering
\begin{tabular}[t]{l|r}
\hline
Type & Count\\
\hline
Quantitative & 10\\
\hline
Qualitative & 23\\
\hline
Date/Time & 4\\
\hline
Boolean & 2\\
\hline
\end{tabular}
\end{table}

\begin{table}[H]
\centering
\caption{List of Variables with Descriptions}
\centering
\begin{tabular}[t]{l|l|l}
\hline
Type & Variable & Description\\
\hline
Quantitative & costo\_x\_perdida & Cost associated with vehicle downtime\\
\hline
Quantitative & num\_max\_viajes & Maximum number of trips made\\
\hline
Quantitative & num\_min\_viajes & Minimum number of trips made\\
\hline
Quantitative & total\_usuarios\_por\_ruta & Total number of users per route\\
\hline
Quantitative & retrazo & Delay time in minutes or hours\\
\hline
Quantitative & dia\_varados & Day when the vehicle was stranded\\
\hline
Quantitative & num\_min\_pasajeros & Minimum number of passengers\\
\hline
Quantitative & num\_max\_pasajeros & Maximum number of passengers\\
\hline
Quantitative & costo\_opot\_min & Cost of downtime (minimum estimate)\\
\hline
Quantitative & costo\_opot\_max & Cost of downtime (maximum estimate)\\
\hline
Qualitative & sistema\_reportado & Report System by the operating area\\
\hline
Qualitative & ruta & Route number or identifier\\
\hline
Qualitative & ubicacion & Location where the failure occurred\\
\hline
Qualitative & vehiculo & Vehicle identifier or license plate\\
\hline
Qualitative & tipologia & Category or typology of the failure\\
\hline
Qualitative & kilometraje & Mileage at the time of the failure\\
\hline
Qualitative & nombre\_de\_conductor & Name of the vehicle operator\\
\hline
Qualitative & hora\_novedad & Time of the incident\\
\hline
Qualitative & observacion\_de\_la\_novedad & Details or notes about the incident\\
\hline
Qualitative & decision & Decision taken regarding the failure\\
\hline
Qualitative & ot & Work order associated with the failure\\
\hline
Qualitative & tenico\_responsable & Responsible technician\\
\hline
Qualitative & sistema & System involved in the failure\\
\hline
Qualitative & subsistema & Subsystem involved in the failure\\
\hline
Qualitative & ref\_comp & Reference component causing the failure\\
\hline
Qualitative & componente & Component affected\\
\hline
Qualitative & adjetivo & Adjective describing the failure\\
\hline
Qualitative & consecuencia & Consequence of the failure\\
\hline
Qualitative & estado & State or status of the failure resolution\\
\hline
Qualitative & observacion & Additional observations\\
\hline
Qualitative & dia\_habil & Whether the day was a business day or not\\
\hline
Qualitative & mes\_falla & Month in which the failure occurred\\
\hline
Qualitative & nueva\_hora & Time when it resumes operation.\\
\hline
Date/Time & fecha & Date when the failure occurred\\
\hline
Date/Time & fecha\_hora\_retrazo & Datetime of the delay caused by the failure\\
\hline
Date/Time & nueva\_fecha & New date assigned after the failure\\
\hline
Date/Time & year\_falla & Year in which the failure occurred\\
\hline
Boolean & varado & Whether the vehicle was stranded (Yes/No)\\
\hline
Boolean & afectacion\_al\_usuario & Whether the user was affected (Yes/No)\\
\hline
\end{tabular}
\end{table}

\subsection{3. Materials and Methods}\label{materials-and-methods}

\subsection{4. Results and Discussion}\label{results-and-discussion}

\subsection{5. Conclusion and
Recommendations}\label{conclusion-and-recommendations}

\subsection{Acknowledgments}\label{acknowledgments}

\subsection{Appendix A. Dataset link}\label{appendix-a.-dataset-link}

\newpage

\begin{thebibliography}{01}
\bibitem{jorge} Referencia n\'umero uno.
\bibitem{2} Referencia n\'umero dos.
\bibitem{3} Referencia n\'umero tres.
\bibitem{Baz} \textsc{Bazaraa, M.S., J.J. Jarvis} y \textsc{H.D. Sherali},
\textit{Programaci\'on lineal y flujo en redes}, segunda edici\'on,Limusa, M\'exico, DF, 2004.
\end{thebibliography}




\end{document}
